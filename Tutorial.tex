\documentclass{unswmaths}
\usepackage[a4paper]{geometry}
\usepackage{fancyhdr}
\pagestyle{fancy}
\begin{document}

\setlength\parindent{0pt}

\fancyfoot[l]{Matthew Kwan, Edward Lee, Adam Gray}
\fancyfoot[r]{\today}
\fancyhead[l]{The University of New South Wales}
\fancyhead[r]{Number Theory - Tutorial 2}

\section*{Question 5}
Let $ A $ denote the set of all integers of the form, $ 2^r3^s5^t $.
Evaluate $ \sum_{n \in A} \frac{1}{n} $ and $ \sum_{n\in A} $.
\subsection*{Solution}
Let $ \mathbb{N}^0 $ denote the set of natural numbers including 0.
We can clearly write
\begin{align*}
	\sum_{n  \in A} \frac{1}{n} &= \sum_{r,s,t \in \mathbb{N}^0} \frac{1}{2^r3^s5^t}. \\
\end{align*}
Now we have that $ \sum_{k=0}^\infty |\ell^{-k}| $ converges for all $ \ell > 1 $,
so we can write
\begin{align*}
	\sum_{r,s,t \in \mathbb{N}^0} \frac{1}{2^r3^s5^t} &= 
	\left( \sum_{r \in \mathbb{N}^0} \frac{1}{2^r} \right)
	\left( \sum_{s \in \mathbb{N}^0} \frac{1}{3^s} \right)
	\left( \sum_{t \in \mathbb{N}^0} \frac{1}{5^t} \right) \\
	&= \left( \frac{1}{1  -\frac{1}{2}} \right) \left( \frac{1}{1 - \frac{1}{3}} \right) \left( \frac{1}{1-\frac{1}{5}} \right) \\
	&= \frac{15}{4}.
\end{align*}
In a similar manner we get that 
\begin{align*}
	\sum_{n  \in A} \frac{1}{n^2} 
		&= \sum_{r,s,t \in \mathbb{N}^0} \frac{1}{\left(2^r3^s5^t\right)^2} \\
		&= \sum_{r,s,t \in \mathbb{N}^0} \frac{1}{4^r9^s25^t} \\
		&=
			\left( \sum_{r \in \mathbb{N}^0} \frac{1}{4^r} \right)
			\left( \sum_{s \in \mathbb{N}^0} \frac{1}{9^s} \right)
			\left( \sum_{t \in \mathbb{N}^0} \frac{1}{25^t} \right) \\
		&= \left( \frac{1}{1 - \frac{1}{4}} \right) \left( \frac{1}{1 - \frac{1}{9}} \right) \left( \frac{1}{1 - \frac{1}{25}} \right) \\
		&= \frac{25}{16}.
\end{align*}

\section*{Question 6}
A slightly generalised version of Bertrand's postulate states that, for
$ n \geq 6 $, there are at least 2 primes between $ n $ and $ 2n $. 
Use this to prove that $ p_{k+2} \leq p_k + p_{k+1} $.
\subsection*{Solution}
By this extended Bertrand's postulate we have that
\begin{align*} 
	p_{k+2} &< 2p_k \\
		&< p_{k+1} + p_k \\
\end{align*}
We just have to tidy up cases when $ p_k < 6 $.
See that 
\begin{align*}
	2 + 3 &\leq 5 \\
	3 + 5 &< 11 \\
	5 + 7 &< 13
\end{align*}

So we have that $ p_{k+2} \leq p_k + p_{k+1} $ which can be strengthened
to $ p_{k+2} < p_k + p_{k+1} $ when for $ k > 1 $. \qed

\section*{Question 7}
\subsubsection*{a} Use Bertrand's Postulate to prove that for 
$ m \geq 2 $, if $ m! = p_1^{\alpha_1} \ldots p_r^{\alpha_r} $ then $ \alpha_i = 1 $  
for at least one value of $ i $.
\subsubsection*{b} Deduce that $ m! $ is never a $ kth $ power 
for any $ k \geq 2 $.

\subsection*{Solution}
\subsubsection*{a}
Write $ m = 2k + 1 $ or $ m = 2k $ for some $ k \in \mathbb{N} $.
In either case there must exist a prime $ k < p < 2k $. 
Just considering $ m > 4 $ we have that in the factorization 
$ m! = m\cdot(m-1)\cdot(m-2)\cdots p \cdots 2 $ there is only 
one power of $ p $ because for $ p > 2 $ we have that
$ p^2 > 2p > m > p $. 
In the language used above, that is simply to say that there 
exists an $i $ such that $ \alpha_i = 1 $.

To tidy up the remaining cases it is clear to see that
\begin{align*}
	4! &= 1 \cdot 2 \cdot 3 \cdot 2^2 \\
	3! &= 1 \cdot 2 \cdot 3 \\
	2! &= 1 \cdot 2 \\
	1! &= 1 \\
\end{align*}
and so the result holds.
\qed
\subsubsection*{b}
It is clear that for any number of the form $ s = x^k $ with $ x, k \in \mathbb{N} $,  $ s $ must have a prime
factorization with $ s = p_1^{k\alpha_1} \ldots p_n^{k\alpha_n} $. For $ m! $, however, we have proven that there are prime factors which
only occur once (not $k > 1 $ times). This is to day $ m! $ cannot be writen in the form
$ x^k $. \qed

\section*{Question 8}
Use Bertrand's postulate to prove that for every integer $ k > 2 $, 
there is a prime p such that $ p < k < 2p $.
\subsection*{Solution}
Consider $ r = \left\lfloor \frac{k}{2} \right\rfloor $ with $ k > 2 $.
We know that there must exist a prime $ p $ such that $ r < p < 2r $. 
Then we have that $ 2r < 2p $. 

If $ k = 2r $ then we are finished. 

If $ k = 2r + 1 $ then $ p \geq r + 1 $ so $ 2p \geq 2r + 2 $ which implies $ 2p > 2r + 1 $.

In conclusion we have that for all $ k > 2 $ there exists a prime $ p $ such that 
$ p < k < 2p $. \qed

\end{document}


