\documentclass{unswmaths}
\begin{document}
\author{Adam J. Gray}
\studentno{3329798}
\subject{Number Theory}
\title{Assignment}

\unswtitle

\section*{Question 1}
Write $ \omega(n) $ for the number of (distinct) prime divisors of $ n $, 
$ \Omega(n) $ for the number of prime factors of $ n $, counted with 
repetition. Thus if, $ n = \prod_{j=1}^m p_j^{k_j} $, then $ \omega(n) = m $,
and $ \Omega(n) = \sum_{j=1}^{m} k_j $.
\subsubsection*{Part a}
Prove that $ 2^{\omega(n)} \leq \tau(n) \leq 2^{\Omega(n)} \leq n $ for 
$ n \geq 2 $.
\subsubsection*{Part b}
When does $ \tau(n) = 2^{\omega(n)} $.
\subsubsection*{Solution}
\subsubsection*{Part a}
Firstly we prove that $ 2^\omega(n) \leq \tau(n) $. 
From the lecture notes we have that if $ n = \prod_{j=1}^m p_j^{k_j} $
then $ \tau(n) = \prod^{m}_{j=1}(k_j + 1) $ so we can say
\begin{align}
	\tau(n) 	&= \prod^{m}_{j=1}\underbrace{(k_j + 1)}_{\geq 2} \nonumber \\
				\label{ineq:qn1a}
				&\leq \prod^{m}_{j=1} 2 \\
				&= 2^m \nonumber \\
				&= 2^{\omega(n)} \nonumber
\end{align}
so $ 2^{\omega(n)} \leq \tau(n) $.

We now show that $ \tau(n) \leq 2^{\Omega(n)} $.
See that 
\begin{align*}
	2^{\Omega(n)} 	&= 2^{\sum_{j=1}^{m} k_j} \\
					&= \prod_{j=1}^{m}2^{k_j}
\end{align*}
and because for all $ k_j \geq 1 $, $ k_j + 1 \leq 2^{k_j} $.
then
\begin{align*}
	\prod_{j=1}^{m}2^{k_j}	\geq \prod_{j=1}^{m} (k_j + 1) \\
\end{align*}
and thus 
$$
	\tau(n) \leq 2^{\Omega(n)}.
$$
It remains to show that $ 2^{\Omega(n)} \leq n $.
Because we have that $$ n = \prod_{j=1}^m p_j^{k_j} $$ and $$ 2^{\Omega(n)} = \prod_{j=1}^{m}2^{k_j} $$
then it is clear because $ 2 \leq k_j $ for all $ j $. 

So we have shown that $ 2^{\omega(n)} \leq \tau(n) \leq 2^{\Omega(n)} \leq n $.
\end{document}

