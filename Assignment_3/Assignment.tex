\documentclass{unswmaths}
\usepackage{color}
\usepackage{unswshortcuts}
\usepackage[a5paper]{geometry}
\begin{document}
\author{Adam J. Gray}
\studentno{3329798}
\subject{Number Theory}
\title{Assignment}

\unswtitle

\section*{Question 1}
Write $ \omega(n) $ for the number of (distinct) prime divisors of $ n $, 
$ \Omega(n) $ for the number of prime factors of $ n $, counted with 
repetition. Thus if, $ n = \prod_{j=1}^m p_j^{k_j} $, then $ \omega(n) = m $,
and $ \Omega(n) = \sum_{j=1}^{m} k_j $.
\subsubsection*{Part a}
Prove that $ 2^{\omega(n)} \leq \tau(n) \leq 2^{\Omega(n)} \leq n $ for 
$ n \geq 2 $.
\subsubsection*{Part b}
When does $ \tau(n) = 2^{\omega(n)} $.

\hrulefill

\subsection*{\textcolor{blue}{Solution}}
\subsubsection*{\textcolor{blue}{Part b}}
Firstly we prove that $ 2^\omega(n) \leq \tau(n) $. 
From the lecture notes we have that if $ n = \prod_{j=1}^m p_j^{k_j} $
then $ \tau(n) = \prod^{m}_{j=1}(k_j + 1) $ so we can say
\begin{align}
	\tau(n) 	&= \prod^{m}_{j=1}\underbrace{(k_j + 1)}_{\geq 2} \nonumber \\
				\label{ineq:qn1a}
				&\leq \prod^{m}_{j=1} 2 \\
				&= 2^m \nonumber \\
				&= 2^{\omega(n)} \nonumber
\end{align}
so $ 2^{\omega(n)} \leq \tau(n) $.

We now show that $ \tau(n) \leq 2^{\Omega(n)} $.
See that 
\begin{align*}
	2^{\Omega(n)} 	&= 2^{\sum_{j=1}^{m} k_j} \\
					&= \prod_{j=1}^{m}2^{k_j}
\end{align*}
and because for all $ k_j \geq 1 $, $ k_j + 1 \leq 2^{k_j} $.
then
\begin{align*}
	\prod_{j=1}^{m}2^{k_j}	\geq \prod_{j=1}^{m} (k_j + 1) \\
\end{align*}
and thus 
$$
	\tau(n) \leq 2^{\Omega(n)}.
$$
It remains to show that $ 2^{\Omega(n)} \leq n $.
Because we have that $$ n = \prod_{j=1}^m p_j^{k_j} $$ and $$ 2^{\Omega(n)} = \prod_{j=1}^{m}2^{k_j} $$
then it is clear because $ 2 \leq k_j $ for all $ j $. 

So we have shown that $ 2^{\omega(n)} \leq \tau(n) \leq 2^{\Omega(n)} \leq n $.

\subsubsection*{\textcolor{blue}{Part a}}
Clearly just when $ n $ is square-free.

\hrulefill

\section*{Question 2}
Define the \textbf{Jordan totient function} by 
$$
    n^k\prod_{p|n}(1-p^-k),
$$
where, as usual, the product is taken over the primes. This is a generalization of Euler's
totient function.
\subsubsection*{Part a}
Prove that $ J $ is multiplicative.
\subsubsection*{Part b}
Show that $ J_k(n) = \sum_{d|n} \mu(d) (\frac{n}{d})^k. $
\subsubsection*{Part c}
Find a simple expression (as a product over primes) for $ J^{-1}(n) $.

\hrulefill
\subsection*{\textcolor{blue}{Solution}}
\subsubsection*{\textcolor{blue}{Part a}}
For $ (n,m) = 1 $ see that
$$
    J_k(nm) = (nm)^k \prod_{p|nm} (1-p^{-k})
$$
but as $ (n,m) = 1 $ then $ p | n $ or $ p | m $ but not both so
\begin{align*}
    J_k(nm) &= n^k m^k \prod_{p|n} (1-p^{-k})\prod_{p|m} (1-p^{-k}) \\
        &= \left( n^k \prod{p|n}(1-p^{-k}) \right)\left( n^k \prod{q|m}(1-q^{-k}) \right) \\
        &= J_k(n) J_k(m)
\end{align*}
\qed

\subsubsection*{\textcolor{blue}{Part b}}
Let $ p $ be a prime and observe that
$$
    J_k(p^\alpha) = p^{\alpha k }(1- p^{-k})
$$
and note in addition that $ J_k(1) = 1 $.
Now define 
$$
    \gamma_k(n) := \sum_{d|n} J_k(d) 
$$
and note that since $ J_k $ is multiplicative, so is $ \gamma_k $.

Therefore we only need to consider $ \gamma_k(p^\alpha) $, where $ p $ is prime,
to characterize $ \gamma_k $.

Therefore observe that
\begin{align*}
    \gamma_k(p^\alpha) &= \sum_{j = 0}^\alpha J_k(p^\alpha) \\
        &= \sum_{j = 1}^\alpha p^{jk}(1-p^k) + 1 \\
        &= (1-p^{-k}) \sum_{j=1}^\alpha p^{jk} + 1 \\
        &= ( 1 - p^{-k})\left( \frac{p^k - p^{k(\alpha + 1)}}{1 - p^k} \right) + 1 \\
        &= p^{\alpha k}.
\end{align*}
So if $ n = p_1^{\alpha_1} \ldots p_r^{\alpha_r} $ then
\begin{align*}
    \gamma_k(n) &= \gamma_k(p_1^{\alpha_1}) \ldots \gamma_k(p_r^{\alpha_r}) \\
        &= p_1^{\alpha_1 k} \ldots p_r^{\alpha_r k} \\
        &= n^k.
\end{align*}
So we have that
\begin{align*}
    \sum_{d|n} J_k(d) &= n^k
\end{align*}
and hence by Mobius inversion
\begin{align*}
    J_k(d) &= \sum_{d|n} \mu(d) \left(\frac{n}{d}\right)^k.
\end{align*}
\qed
\subsubsection*{\textcolor{blue}{Part c}}

Let $ N^k(n) = n^k $. We can then write that $ J_k =  N^k * \mu $. 
So
\begin{align*}
	J_k^{-1} &= (N^k * \mu )^{-1} \\
	&= \mu^{-1} * N^{k -1}. \\
\end{align*}
As it is clear that $ N_k $ is completely multiplicative so $ N^{k -1}(n) = \mu(n)N^{k} $,
and hence 
\begin{align*}
	J_k^{-1}(n) &= \sum_{d|n} u(d) \mu\left( \frac{n}{d} \right) N^{k}\left( \frac{n}{d} \right) \\
		&= \sum_{d|n} \mu(d) N^{k}(d)
\end{align*}
Now as $ \mu N^{k} $ is multiplicative $ J_k^{-1} $ is the Dirichlet product of two multiplicative functions,
and is hence multiplicative.
For $ p $ prime we have that
\begin{align*}
	J_k^{-1}(p^\alpha) &= \sum_{j=0}^{\alpha}\mu(p^j)p^{jk} \\
		&= (1 - p^k)
\end{align*}

So if $ n = p_1^{\alpha_1} p_2^{\alpha_2} \ldots p_r^{\alpha_r} $ where the $ p_j $ are distinct primes, then
$$
	J_{k}^{-1}(n) = \prod_{j = 1}^{r} \left( 1 - p^k \right).
$$
\qed

\hrulefill

\section*{Question 3}

Write $ M_2(x) = \sum{ n \leq x } (\mu(n))^2 $.

Hence $ M_2(x) $ counts the number of square free integers $ \leq x $.

\subsubsection*{Part a}
Explain why $ (\mu(n))^2 = \sum_{m^2 | n} \mu(m) $.

\subsubsection*{Part b}
Prove that 
$$ 
    M_2(x) = x \sum_{m \leq \sqrt{x}} \frac{\mu(n)}{m^2} - \sum_{m \leq \sqrt{x}} \mu(m) \left\{ \frac{x}{m^2} \right\}.
$$    

\subsubsection*{Part c}
Deduce that 
$$
	M_2(x) = \frac{6}{\pi^2}x + O(\sqrt{x}).
$$

\subsubsection*{Part d}
Interpret this result in terms of the proportion of square-free numbers
in the interval $[ 1, x] $.
\subsubsection*{Part e}
Use MAPLE (or otherwise) to count the number of square-free numbers between $ 1 $ and $ 1000 $. (Comment!)

\hrulefill

\subsection*{\textcolor{blue}{Solution}}

\subsubsection*{\textcolor{blue}{Part a}}
Write $ n = p_1^{\alpha_1} p_2^{\alpha_2} \ldots p_r^{\alpha_r} $ where $ p_j $ 
is prime and all $ p_j $ are distinct.
Without loss of generality assume we can reorder this factorization such that 
there exists a $ 1 \leq t \leq r $ such that
when $ j < t $, $ 2 \leq \alpha_j $.
Then it can be easily seen that
\begin{align*}
    \sum_{m^2 | n } \mu(m) &= 1 + \mu(p_1) + \mu(p_2) + \ldots + \mu(p_{t-1}) \\
        &+ \mu(p_1 p_2) + \mu(p_1 p_3) + \ldots + \mu(p_{t-2} p_{t-1}) \\
        &+ \mu(p_1 p_2 p_3) + \ldots + \mu( p_{t-3} p_{t-2} p_{t-1}) \\
        &+ \ldots \\
        &+ \mu(p_1 p_2 \ldots p_{t-1}) \\
        &=  \binom{t-1}{0}(-1)^{0} + \binom{t-1}{1} (-1)^{1} + \ldots + \binom{t-1}{t-1}(-1)^{t-1} \\
        &= \begin{cases}
               (1-1)^{t-1} = 0 & \text{ where } n \text{ is not square-free } \\
               1 & \text{ where } n \text{ is square-free }
           \end{cases}
\end{align*}
It should be noted that higher powers of primes do not appear in the sum
because they are not square-free and hence Mobius function would be $ 0 $ 
for these numbers.

Since it is clear that 
\begin{align*}
    (\mu(n))^2 = \begin{cases}
        (1-1)^{t-1} = 0 & \text{ where } n \text{ is not square-free } \\
        1 & \text{ where } n \text{ is square-free }
    \end{cases}
\end{align*}
we have shown that $ (\mu(n))^2 = \sum_{m^2 | n} \mu(m) $. \qed

\subsubsection*{\textcolor{blue}{Part b}}
If we prove two simple lemmas (which are just DSIs), the solution to this question is very straight-forward. 

\begin{lemma}
\label{lem:1}
If $ f $ and $ g $ are arithmetic functions then
$$
	\sum_{n\leq x} \sum_{m^2 |n} = \sum_{m\leq \sqrt{x}} g(m) \sum_{j \leq \frac{x}{m^2}} f(m^2j).
$$
\end{lemma}
\begin{proof}
	See that
	\begin{align*}
		\sum_{n\leq x}f(n) \sum_{m^2|n} g(m) &= f(1)g(1) + f(2)g(1) + \ldots \\
		&+ f(4)[ g(1) + g(2)] + f(5)g(1) + \ldots \\  
		&+ f(8)[g(1) + g(2)] + \ldots \\
			&\text{ by reordering } \\
			&= g(1)[ f(1) + f(2) + f(3) + \ldots ] + \\
				&+ g(2)[ f(4) + f(8) + f(12) + \ldots ] + \ldots \\
			&= \sum_{m \leq \sqrt{x}} g(m) \sum_{jm^2 \leq x}f(jm^2).
	\end{align*}
\end{proof}

\begin{lemma}
	\label{lem:2}
	If $ g $ is an arithmetic function then 
	$$
 		\sum_{n\leq x} \sum_{m^2|n} g(m) = \sum_{m\leq \sqrt{x}} g(m) \left\lfloor \frac{x}{m^2} \right\rfloor.
	$$
\end{lemma}
\begin{proof}	
	Setting $ f = u $ in lemma \ref{lem:1} yields
	\begin{align*}
		\sum_{n \leq x} \sum_{m^2|n} g(m) &= \sum_{n\leq \sqrt{x}} u(n) \sum_{m^2|n} g(m) \\
			&= \sum_{m \leq \sqrt{x}} g(m) \sum_{jm^2 \leq x} u(m)
	\end{align*}
	and since it is clear to see that 
	\begin{align*}
		\sum_{jm^2 \leq x} u(m) &= \sum_{j \leq \frac{x}{m^2}} u(m)  \\
			&= \left\lfloor \frac{x}{m^2} \right\rfloor 
	\end{align*}
	it follows that 
	$$
		\sum_{n\leq x} \sum_{m^2|n} g(m) = \sum_{m\leq \sqrt{x}} g(m) \left\lfloor \frac{x}{m^2} \right\rfloor.
	$$
\end{proof}

We now return to the original problem.

Letting $ g = \mu $ in lemma \ref{lem:2} we get that 
$$
	\sum_{n\leq x} \sum_{m^2 | n} \mu(m) = \sum_{m \leq{\sqrt{x}}} \mu(m) \left\lfloor \frac{x}{m^2} \right\rfloor.
$$
Its then clear to see that
\begin{align*}
	\sum_{m\leq \sqrt{x}} \mu(x) \left\lfloor \frac{x}{m^2} \right\rfloor &= \sum_{m \leq \sqrt{x}} \mu(x) \left( \frac{x}{m^2} - \left\{ \frac{x}{m^2} \right\} \right)
		&= x \sum_{m \leq \sqrt{x}} \frac{\mu(m)}{m^2} - \sum_{m \leq \sqrt{x}} \mu(m) \left\{ \frac{x}{m^2} \right\}
\end{align*}
So we have shown that
$$ 
    M_2(x) = x \sum_{m \leq \sqrt{x}} \frac{\mu(n)}{m^2} - \sum_{m \leq \sqrt{x}} \mu(m) \left\{ \frac{x}{m^2} \right\}. 
$$ \qed

\subsubsection*{\textcolor{blue}{Part c}}

Since $ \zeta(2) = \frac{\pi^2}{6} $ and 
$$
	\sum_{m=1}^{\infty} \frac{\mu(m)}{m^2} = \frac{6}{\pi^2}
$$
then 
\begin{align*}
	x \sum_{m \leq \sqrt{x}} \frac{\mu(m)}{m^2} &= x \frac{6}{\pi^2} - x \left( \sum_{m > \sqrt{x}} \frac{\mu(m)}{m^2} \right) \\
		&= x\frac{6}{\pi^2} + xO\left( \sum_{m>\sqrt{x}} \frac{1}{m^2} \right) \\
		&= x\frac{6}{\pi^2} + xO\left( \frac{1}{\sqrt{x}} \right) \\
		&= x\frac{6}{\pi^2} + O\left( \sqrt{x} \right) \\
\end{align*}

Additionally it is clear that 
$$
O\left(\sum_{m \leq \sqrt{x}} \mu(m) \left\{ \frac{x}{m^2} \right\}\right) = O(1)
$$
and so 
$$
	M_2(x) = \frac{6}{\pi^2} + O(\sqrt{x}).
$$
\qed

\subsubsection*{\textcolor{blue}{Part d}}
There are $ x $ natural numbers in the interval  $ [1, x] $ so the proportion of square-free integers in the interval 
$ [1, x] $ would be given by 
$$
	\frac{M_2(x)}{x}
$$
which is equal to 
$$
	\frac{6}{\pi^2} + O\left(\frac{1}{\sqrt{x}}\right).
$$
So as $ x \lra \infty $ then the proportion of square free integers in the interval becomes ``closer'' to 
$ \frac{6}{\pi^2} $.
\subsubsection*{\textcolor{blue}{Part e}}

There are 608 square free integers between $ 1 $ and $ 1000 $ and $ \frac{6}{\pi^2} \times 1000 \approx 607.9 $
which is a very close approximation.
\end{document}


