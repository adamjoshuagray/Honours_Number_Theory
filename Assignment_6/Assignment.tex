\documentclass{unswmaths}
\usepackage{color}
\usepackage{unswshortcuts}
\begin{document}
\author{Adam J. Gray}
\studentno{3329798}
\subject{Number Theory}
\title{Assignment}

\unswtitle

\section*{Question 1}

Suppose $ x > 2 $ and let $ m $ be the largest integer such that $ 2^m \leq x $.

\subsection*{Part a}
Use the definition of $ \psi(x) $ to deduce that $ \psi(x) \geq \vartheta(x) $ and conclude from Tutorial problem 1 that
$ \vartheta(x) \leq 2x $.

\subsection*{Part c}

Show that $ \frac{\log(x) }{x^\alpha} $ has a maximum of $ \frac{1}{\alpha e} $

\subsection*{Part d}

Deduce that $ \\psi(x) - \vartheta(x) \leq 9x^\frac{1}{2} $.

\subsection*{Part e}

Conclude that, as $ x \lra \infty $, $ \frac{\psi(x)}{x} \lra 1 \Leftrightarrow \frac{\vartheta(x)}{x} \lra 1 $.

\subsubsection*{Solution}

\subsubsection*{Part a}

We have that 

\begin{align*}
	\psi(x) &= \sum_{m \leq \log_2(x)} \vartheta(x^\frac{1}{m}) \\
		&= \vartheta(x) + \underbrace{\sum_{2 \leq m \leq \log_2(x)} \vartheta(x^\frac{1}{m})}_{\geq 0} \\
\end{align*}
so it is obvious that $ \psi(x) \geq \vartheta(x) $.

From the tutorial problems we have that $ \psi(x) \leq 2x $ so it must also be that $ \vartheta(x) \leq 2x $.


\subsubsection*{Part b}

See that 
\begin{align*}
	\frac{d}{dx} \frac{log(x)}{x^\alpha} &= \frac{\frac{1}{x} x^\alpha - \alpha \log(x) x^{\alpha-1}}{x^{2\alpha}} \\
		&= \underbrace{\frac{x^{\alpha-1}( 1 - \alpha \log(x)}{x^{2 \alpha}}}_{\circledast}.
\end{align*}
Now setting $ \circledast = 0 $
we have that 

\begin{align*}
	\log(x) &= \frac{1}{\alpha} \\
	x &= e^{\frac{1}{\alpha}}.
\end{align*}


Now checking
\begin{align*}
	\frac{d^2}{dx^2} \frac{\log(x)}{x^\alpha} \Big|_{x = e^{\frac{1}{\alpha}}} &= e^{\frac{-2-\alpha}{2}}(1-\alpha)(1 - 1) + e^\frac{-1-\alpha}{\alpha}\left( \frac{-\alpha}{e^{-\alpha}} \right) \\
	&\leq 0 
\end{align*}
so there must be a maximum at $ x = e^\frac{1}{\alpha} $. Then evaluating we get that
$$
	\frac{\log(x)}{x^\alpha} \Big|_{x = e^\frac{1}{\alpha}} = \frac{1}{e\alpha}.
$$

\qed

\subsubsection*{Part c}

We can write
\begin{align*}
	\psi(x) - \vartheta(x) &= \sum_{1 \leq m \leq \log_2(x)} \vartheta(x^\frac{1}{m}) - \vartheta(x) \\
		&= \sum_{2 \leq m \leq \log_2(x)} \vartheta(x^\frac{1}{m}).
\end{align*}
Now using the result of \textbf{part a} we can write

\begin{align*}
	\sum_{2 \leq m \leq \log_2(x)} \vartheta(x^\frac{1}{m}) &\leq \sum_{2 \leq m \leq \log_2(x)} 2x^\frac{1}{m} \\
		&\leq 2x^\frac{1}{2} + \sum_{3 \leq m \leq \log_2(x)} 2x^\frac{1}{m} \\
		&\leq 2x^\frac{1}{2} + \sum_{3 \leq m \leq \log_2(x)} 2x^\frac{1}{3} \\
		&\leq 2x^\frac{1}{2} + \frac{2}{\log(2)} \frac{\log(x)}{x^\frac{1}{6}} x^\frac{1}{2}.
\end{align*}

Now by applying the result of \textbf{part b} we can see that
$$
	\frac{\log(x)}{x^\frac{1}{6}} \leq \frac{6}{e}
$$
so we have that 
\begin{align*}
	2x^\frac{1}{2} + \frac{2}{\log(2)} \frac{\log(x)}{x^\frac{1}{6}} x^\frac{1}{2} \leq x^\frac{1}{2} \left( 2 + \frac{2}{\log(2)} \frac{6}{e} \right).
\end{align*}
Now by numerical evaluation we see that 
$$
	\left( 2 + \frac{2}{\log(2)} \frac{6}{e} \right) < 9
$$
and so 
$$
	\psi(x) - \vartheta(x) \leq 9x^\frac{1}{2}.
$$
\qed
\subsubsection*{Part d}
By exploiting the result of \textbf{part c} we see that
\begin{align*}
	0 \leq \lim_{x \lra \infty} \left( \frac{\psi(x)}{x} - \frac{\vartheta(x)}{x}\right) &\leq \lim_{x \lra \infty} \frac{9x^\frac{1}{2}}{x} \\
		&= 0
\end{align*}
so 
$$
	\lim_{x \lra \infty} \frac{\psi(x)}{x} = \lim_{x \lra \infty} \frac{\vartheta(x)}{x}
$$
which means that
$$
	\lim_{x \lra \infty} \frac{\psi(x)}{x} = 1 \Leftrightarrow \lim_{x \lra \infty} \frac{\vartheta(x)}{x} = 1.
$$
\qed
\section*{Question 2}

\subsection*{Part a}
Assuming that 
$$
	\lim_{x \lra \infty} \frac{\pi(x)\log(x)}{x} = 1
$$
show that
$$
	\lim_{x \lra \infty} \frac{\log(\pi(x))}{\log(x)} = 1.
$$
\subsection*{Part b}
Deduce that 
$$
	\lim_{x \lra \infty} \frac{\pi(x)\log(pi(x))}{x} = 1.
$$

\subsection*{Part c}
If $ p_n $ denotes that $n$th prime, show that the PNT implies 
$$
	\lim_{n \lra \infty } \frac{n \log(n)}{p_n} = 1.
$$
(This says that the $n$th primes is `roughly' $ n\log(n) $ for large $n$.)

\subsubsection*{Solutions}
\subsubsection*{Part a}
Given
$$
	\lim_{x \lra \infty} \frac{\pi(x)\log(x)}{x} = 1
$$
we take the $ log $ of both sides to get
$$
	\lim_{x \lra \infty} \log(\pi(x)) + \log(\log(x)) - \log(x) = 0
$$
and dividing by $ \log(x) $ we get that
$$
	\lim_{x \lra \infty} \frac{\log(\pi(x)) + \log(\log(x)) - \log(x)}{\log(x)} = 0.
$$

Now see that
\begin{align*}
	\lim_{x \lra \infty} \frac{\log(\log(x))}{\log(x)} &= \lim_{x \lra \infty} \frac{\frac{1}{x \log(x)}}{\frac{1}{x}} \\
		&= \lim_{x \lra \infty} \frac{1}{\log(x)} \\
		&= 0.
\end{align*}

and 

\begin{align*}
	\lim_{x \lra \infty} \frac{\log(\pi(x)) + \log(\log(x)) - \log(x)}{\log(x)} &= \lim_{x \lra \infty} \frac{\log(\pi(x))}{\log(x)} - 1 \\
		&= 0
\end{align*}
and so the result follows.
\qed
\subsubsection*{Part b}

The assumption of \textbf{part a} is essentially that

$$
	\pi(x) \sim \frac{x}{\log(x)}
$$
or equivalently 
$$
	\frac{\pi(x)}{x} \sim \frac{1}{\log(x)}
$$

so we can deduce that from \textbf{part a} that
\begin{align*}
	\lim_{x \lra \infty} \frac{\log(\pi(x))}{\log(x)} &= \lim_{x \lra \infty} \frac{\pi(x) \log(\pi(x))}{x} \\
		\text{ and so } \\
	\lim_{x \lra \infty} \frac{\pi(x) \log(\pi(x))}{x} &= 1.
\end{align*}
\qed
\subsubsection*{Part c}
The prime number theorem asserts that the assumption in \textbf{part a} is in fact correct.

We can write the result of \textbf{part b} as
\begin{align*}
	\lim_{p_n \lra \infty} \frac{n \log(n)}{p_n} &= 1
\end{align*}
or equivalently as
\begin{align*}
	\lim_{n \lra \infty} \frac{n \log(n)}{p_n} &= 1
\end{align*}
\qed

\section*{Question 3}

Use the PNT to show heuristically that there should be about $ \pi(n) $ primes between $ n^2 $ and $ (n+1)^2 $.

\subsubsection*{Solution}

The prime number theorem states that
$$
	\pi(x) \sim \frac{x}{\log(x)}
$$
so we can write
$$
	\pi(n^2) \sim \frac{n^2}{2\log(n)}
$$
and
$$
	\pi((n+1)^2) \sim \frac{(n+1)^2}{2\log(n+1)}
$$
so we can say that ``heuristically''
\begin{align*}
	\pi((n+1)^2) - \pi(n^2) \sim  \frac{(n+1)^2}{2\log(n+1)} - \frac{n^2}{2\log(n)}.
\end{align*}
Now as $ \log(n) \sim \log(n+1) $ we can write
$$
	\frac{(n+1)^2}{2\log(n+1)} \sim \frac{(n+1)^2}{2 \log(n)}
$$
and so
\begin{align*}
	\pi((n+1)^2) - \pi(n^2) &\sim  \frac{(n+1)^2}{2\log(n)} - \frac{n^2}{2\log(n)} \\
		&= \frac{2n + 1}{2 \log(n)} \\
		&\sim \frac{n}{\log(n)} \\
		&\sim \pi(n).
\end{align*}
\end{document}


