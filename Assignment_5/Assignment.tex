\documentclass{unswmaths}
\usepackage{color}
\usepackage{unswshortcuts}
\begin{document}
\author{Adam J. Gray}
\studentno{3329798}
\subject{Number Theory}
\title{Assignment}

\unswtitle

\section*{Question 1}
\subsection*{Part a}
Use the comparison test for series to prove that for  $ s > 11 $, (s real),
$$
	\frac{1}{s-1} \leq \zeta(s) \leq 1 + \frac{1}{s-1}.
$$
\subsection*{Part b}
Deduce that $ \lim_{s \lra 1^+} (s-1)\zeta(s) = 1 $.
\subsection*{Part c}
Further show that $ \frac{\log(\zeta(s))}{\log(\frac{1}{s-1})} \lra 1 $ as $ s \lra 1^+ $, and explain what this means in terms of 
the Dirichlet density of the set of all primes.

\subsubsection*{Solution}

\subsubsection*{Part a}

First note that
$$
	\int_1^\infty \frac{1}{t^s} dt = \frac{1}{s-1}.
$$

So we wish to show that 

$$
	\int_1^\infty \frac{1}{t^s} dt \leq \sum_{n=1}^\infty \frac{1}{n^s}
$$

and
$$
	\sum_{n=2}^\infty \frac{1}{n^s} \leq \int_1^\infty \frac{dt}{t^s} 
$$

which is equivelent to showing that

\begin{align}
	\label{ineq-1}
	\sum_{n=1}^\infty \int_{n}^{n+1} \frac{dt}{t^s} \leq \sum_{n=1}^\infty \int_{n-1}^{n} \frac{dt}{\lceil t \rceil^s}
\end{align}

and

\begin{align}
	\label{ineq-2}
	\sum_{n=2}^\infty \int_{n-1}^{n} \frac{dt}{\lceil t \rceil^s} \leq \sum_{n=2}^\infty \int_{n-1}^n \frac{dt}{t^s}.
\end{align}

By the integral estimation lemma we see that each term in \eqref{ineq-1}
$$
	\int_{n}^{n+1} \frac{dt}{t^s} \leq \frac{1}{n^s}
$$
and we also have that
$$
	\int_{n}^{n+1} \frac{dt}{\lceil t \rceil^s} = \frac{1}{n^s}
$$
so clearly 
$$
	\int_{n}^{n+1} \frac{dt}{t^s} \leq \int_{n}^{n+1} \frac{dt}{\lceil t \rceil^s}
$$
and hence \eqref{ineq-1} holds.

In a similar manner for \eqref{ineq-2} we have that 
$$
	\int_{n-1}^n \frac{dt}{t^s} \geq \frac{1}{n^s}
$$
so we have that
$$
	\int_{n-1}^{n} \frac{dt}{\lceil t \rceil^s} \leq \int_{n-1}^n \frac{dt}{t^s}
$$
and \eqref{ineq-2} follows.

\subsection*{Part b}

By multiplying all terms of the result of part a we get that
$$
	1 \leq (s-1)\zeta(s) \leq 1 + (s-1)
$$

and so
$$
	1 \leq \lim_{s \lra 1^+} (s-1)\zeta(s) \leq \lim_{s\lra 1^+} 1 + (s-1)
$$
which immidiatly implies

$$
	\lim_{s \lra 1^+} (s-1) \zeta(s) = 1.
$$

\subsubsection*{Part c}

Taking the log of the result of part a we get that

$$
	\log(\frac{1}{s-1}) \leq \log(\zeta(s)) \leq \log(1 + \frac{1}{s-1}).
$$
which means that
\begin{align}
	\label{ineq-3}
	1 \leq \frac{\log(\zeta(s))}{\log\left( \frac{1}{s-1}\right)} \leq \frac{\log \left( 1 + \frac{1}{s-1} \right)}{\log\left( \frac{1}{s-1}\right)}.
\end{align}

We now evalutate
\begin{align*}
	\lim_{s \lra 1^+} \frac{\log \left( 1 + \frac{1}{s-1} \right)}{\log\left( \frac{1}{s-1}\right)} &= \lim_{s \lra 1^+}\frac{\frac{1}{1+\frac{1}{s-1}}\left( \frac{-1}{(s-1)^2}\right)}{\frac{1}{\frac{1}{s-1}}\left( \frac{-1}{(s-1)^2}\right)} \\
		&= \lim_{s \lra 1^+} \frac{\frac{(s-1)}{s}}{(s-1)} \\
		&= 1
\end{align*}
so by taking limits of \eqref{ineq-3} we get that 

$$
	\lim_{s \lra 1^+} \frac{\log(\zeta(s))}{\log(\frac{1}{s-1})} = 1
$$

This is completely unsuprising as it means that the set of all primes has Dirichlet density 1.
\section*{Question 3}

Let $ M(x) = \sum_{n \leq x} \mu(n) $.

\subsection*{Part a}

Use the fact that 
$$
	\frac{1}{\zeta(s)} = \sum_{n =1}^\infty \frac{\mu(n)}{n^s}
$$
for $ \sigma = \real(s) > 1 $, to deduce that 
$$
	\frac{1}{\zeta(s)} = s \int_1^\infty M(x)x^{-s-1} dx.
$$

\subsection*{Part b}
Assume that $ M(x) = O(x^{\frac{1}{2} + \epsilon}) $, for any $ \epsilon > 0 $ is true and deduce 
that the integral on the right convergess for $ \sigma > \frac{1}{2} + \epsilon $.

\subsection*{Part c}

Hence explain why $ M(x) = O(x^{\frac{1}{2} + \epsilon}) $, for any $ \epsilon > 0 $, implies the Riemann Hypothesis.

\subsubsection*{Solutions}

\subsubsection*{Part a}

Firstly note that
\begin{align*}
	\int_{n}^{n+1} \frac{d}{dx}\left( \frac{1}{x^s} \right) dx &= -\int_{n}^{n+1}\frac{s}{x^{s+1}}dx \\
		&= \left[ \frac{1}{x^{s+1}} \right]_{x = n}^{x=n+1} \\
		&= \left( \frac{1}{(n+1)^{s+1}} - \frac{1}{(n^s)} \right).
\end{align*}

Also note that
\begin{align*}
	\sum_{n=1}^\infty \frac{\mu(n)}{n^s} &= \sum_{n=1}^\infty \frac{M(n) - M(n-1)}{n^s} \\
		&= \sum_{n=1}^\infty M(n) \left( \frac{1}{n^s} - \frac{1}{(n+1)^s} \right) \\
		& \text{ and by using the observation above} \\
		&= \sum_{n=1}^\infty M(n) \int_{n-1}^n \frac{s}{x^{s+1}} dx \\
		&= \sum_{n=1}^\infty \int_{n-1}^n \frac{s M(x)}{x^{s+1}} dx \\
		&= s \int_1^\infty\frac{M(x)}{x^{s+1}} dx
\end{align*}

\subsubsection*{Part b}

Let $ K(x) = \frac{M(x)}{x^{s+1}} $ then
\begin{align*}
	K(x) &= \frac{O(x^{\frac{1}{2} + \epsilon})}{x^{s+1}} \\
		&= O(x^{-s-\frac{1}{2} + \epsilon})
\end{align*}
then we have that 
\begin{align*}
	\int_1^\infty K(x) dx &= \int_1^\infty O(x^{-s-\frac{1}{2} + \epsilon}) dx \\
		&< \infty
\end{align*}
so long as if $ s = \sigma + it $, then $ \sigma > \frac{1}{2} + \epsilon $, by the $ p $ test.



We don't need to consider the imaginary component because, $| x^s | $ is independent
of its imaginary component, i.e. 

$$ |x^{\sigma + it}| = |e^{\ln(x)(\sigma + it)}| = |e^{\ln(x)\sigma}|. $$

\subsubsection*{Part c}
If you can prove that $ M(x) = O(x^{\frac{1}{2} + \epsilon }) $ then the integral converges for all $ \sigma > \frac{1}{2} + \epsilon $
and hence $ \zeta(s) = \zeta(\sigma + it) \neq 0 $ for any choice of $ \epsilon > 0 $. This would mean that the only non-trivial zeros
must have $ \sigma \leq \frac{1}{2} $. The functional equation strengthens this by symetry to say $ \sigma = \frac{1}{2} $ for any non-trivial
zero of the Riemann-Zeta function.

\end{document}


