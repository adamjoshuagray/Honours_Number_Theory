\documentclass{unswmaths}
\usepackage{color}
\usepackage{unswshortcuts}
\begin{document}
\author{Adam J. Gray}
\studentno{3329798}
\subject{Number Theory}
\title{Assignment}

\unswtitle

\section*{Question 1}
\subsection*{Part a}
Use the character table given in lectures for $ \mathbb{Z}_5 $, extended to a Dirichlet character,to evaluate
$$
	\sum_{i=1}^4 \chi_i(n) \overline{ \chi_i(b)}, \ \ \ \text{ for each } b \in \mathbb{U}_5.
$$
\subsection*{Part b}
Use the results of (a) to prove, in detail, that there are infinitely many primes congruent to  $ 1 \mod 5 $,
$ 2 \mod 5 $, and $ 3 \mod 5 $ and $ 4 \mod 5 $.

\subsection*{Solution}
For this question we use the following character table:

\begin{table}[h]
	\centering
	\begin{tabular}{c | c | c | c | c}
		& 1 & 3 & 4 & 2 \\ \hline 
		$\chi_1$ & 1 & 1 & 1 & 1 \\
		\hline
		$\chi_2$ & 1 & -1 & 1 & -1 \\
		\hline
		$\chi_3$ & 1 & i & -1 & -i \\
		\hline
		$\chi_4$ & 1 & -i & -1 & i
	\end{tabular}
\end{table}
	
\subsection*{Part a}
\begin{align*}
	\sum_{i=1}^4 \chi_i(n) \overline{ \chi_i(b)} &=
	\begin{cases}
		0 & \text{ if } n \not\equiv b \mod 5 \\
		4 & \text{ if } n \equiv b \mod 5
	\end{cases}
\end{align*}
This follows immediately from the orthogonality relation proved in lectures. (With the subtle difference that
this is a Dirichlet character, but because $ b \in \{ 1,2,3,4 \} $ this makes no difference.)
\subsection*{Part b}
Firstly see that 
\begin{align*}
	L(s, \chi_i) &= \sum_{n=1}^\infty \frac{\chi_i(n)}{n^s} \\
		&= \prod_{p \text{ prime }} \left( 1 - \frac{\chi_i(p)}{p^s} \right)^{-1}
\end{align*}
and so 
\begin{align*}
	\log L(s, \chi_i) &= - \sum_{n=1}^\infty \log \left( 1 - \frac{\chi_i(p)}{p^s} \right) \\
		&= \sum_{p \text{ prime}} \frac{\chi_i(p)}{p^s} + \frac{1}{2}\left( \frac{\chi_i(p)}{p^s} \right)^2 
			+ \frac{1}{3}\left( \frac{\chi_i(p)}{p^s} \right)^3 + \cdots \\
		&= \sum_{p \text{ prime}} \frac{\chi_i(p)}{p^s} + R_i(s)
\end{align*}
where
\begin{align*}
	R_i(s) &= \sum_{p \text{ prime}} \frac{1}{2}\left( \frac{\chi_i(p)}{p^s} \right)^2 + 
		\frac{1}{3}\left( \frac{\chi_i(p)}{p^s} \right)^3 + \frac{1}{4}\left( \frac{\chi_i(p)}{p^s} \right)^4 + \cdots. \\
\end{align*}
See that
\begin{align*}
	|R_i(s)| &\leq \sum_{p \text{ prime}} \frac{1}{2}\left( \frac{|\chi_i(p)|}{p^s} \right)^2 + 
		\frac{1}{3}\left( \frac{|\chi_i(p)|}{p^s} \right)^3 + \frac{1}{4}\left( \frac{|\chi_i(p)|}{p^s} \right)^4 + \cdots \\
		&\leq \sum_{p \text{ prime}} \frac{1}{2}\left( \frac{1}{p^s} \right)^2 + 
		\frac{1}{2}\left( \frac{1}{p^s} \right)^3 + \frac{1}{2}\left( \frac{1}{p^s} \right)^4 + \cdots \\
		& \text{ by the geometric sum formula } \\
		&= \frac{1}{2} \sum_{p \text{ prime}} \frac{1}{p^s(p^s-1)} \\
		&< \frac{1}{2} \sum_{p \text{ prime}} \frac{1}{p(p-1)} \\
		&< \frac{1}{2} \sum_{n=1}^\infty \frac{1}{n(n-1)} \\
		&= \frac{1}{2}
\end{align*}
and so $ R_i(s) $ is bounded as $ s \longrightarrow 1^{+} $.

Now see that 
\begin{align*}
	L(s,\chi_1) = \sum_{b = 1}^4 \sum_{n=0}^\infty \frac{1}{(5n+b)^s} \longrightarrow \infty \text{ as } s \longrightarrow 1^+.
\end{align*}
For $ \chi_2 $ see that 
\begin{align*}
	\Re(L(s, \chi_2)) &= \underbrace{\frac{1}{1^s} - \frac{1}{2^s} -\frac{1}{3^s} + \frac{1}{4^s}}_{>\frac{1}{6}} + \underbrace{\frac{1}{6^s} - \frac{1}{7^s} + \frac{1}{8^s} - \frac{1}{9^s}}_{> 0 } + \underbrace{\cdots}_{>0} \\
		&> \frac{1}{6}
\end{align*}
and 
\begin{align*}
	\Re(L(s, \chi_2)) &= \frac{1}{1^s} - \underbrace{\left( \frac{1}{2^s} +\frac{1}{3^s} - \frac{1}{4^s} - \frac{1}{6^s} \right)}_{>0} - 
		\underbrace{\left(\frac{1}{7^s} + \frac{1}{8^s} - \frac{1}{9^s} - \frac{1}{11^s} \right)}_{>0} - \underbrace{\cdots}_{>0} \\
		&< 1
\end{align*}
so $ \frac{1}{6} < \Re(L(s, \chi_2)) < 1 $ for all $ s > 1 $.
Also note that
\begin{align*}
	\Im( L(s, \chi_2)) = 0.
\end{align*}
For $ \chi_3 $ see that
\begin{align*}
	\Re(L(s, \chi_3)) = \underbrace{\frac{1}{1^s} - \frac{1}{4^s}}_{>\frac{3}{4}} + 
		\underbrace{\frac{1}{6^s} + \frac{1}{9^s}}_{>0} + \underbrace{\cdots}_{>0} \\
		> \frac{3}{4}
\end{align*}
and 
\begin{align*}
	\Re(L(s, \chi_3)) &= \frac{1}{1^s} - \underbrace{\left( \frac{1}{4^s} - \frac{1}{6^s} \right)}_{>0} - \underbrace{\left( \frac{1}{9} - \frac{1}{11} \right)}_{>0} - \underbrace{\cdots}_{>0} \\
		&< 1
\end{align*}
so $ \frac{3}{4} < \Re(L(s, \chi_3)) < 1 $ for $ s > 0 $. 
We can also see that $$ \Re(L(s, \chi_4)) = \Re(L(s, \chi_3)) $$ and so $ \frac{3}{4} < \Re(L(s, \chi_4)) < 1 $ for $ s > 0 $.

Also note that
\begin{align*}
	\Im(L(s, \chi_3)) &= -\frac{1}{2} + \underbrace{\frac{1}{3} - \frac{1}{7}}_{>0} + \underbrace{\frac{1}{8} - \frac{1}{12}}_{>0} + \underbrace{\cdots}_{>0} \\
		&> -\frac{1}{2}
\end{align*}
and
\begin{align*}
	\Im(L(s, \chi_3)) &= \underbrace{-\frac{1}{2} + \frac{1}{3}}_{<0} + \underbrace{- \frac{1}{7} + \frac{1}{8}}_{<0} + \underbrace{\cdots}_{<0} \\
		&< 0
\end{align*}
so $ -\frac{1}{2} < \Im(L(s, \chi_3)) < 0 $.

Also $ \Im( L(s, \chi_4) ) = - \Im( L(s, \chi_3) ) $ so $ 0 < \Im( L(s, \chi_4) ) < \frac{1}{2} $.

Now see that (by applying the result of part a)
\begin{align*}
	\sum_{i=1}^4 \chi_i(b) \log( L(s, \chi_i)) &= \sum_{p \text{ prime}} \frac{\sum_{i=1}^{4}\chi_i(b)\chi_i(p)}{p^s} + \underbrace{\sum_{i=1}^4 \chi_i(b)R_i(s)}_{\text{ bounded as } s \lra 1^+}\\
		&= 4\sum_{\substack{p \text{ prime} \\ p \equiv b\mod 5 }} \frac{1}{p^s} + \underbrace{\sum_{i=1}^4 \chi_i(b)R_i(s)}_{\text{ bounded as } s \lra 1^+}.
\end{align*}
Now as only $ \log(L(s, \chi_1)) $ is unbounded as $ s \lra 1^+ $ then 
$$
\sum_{\substack{p \text{ prime} \\ p \equiv b\mod 5 }} \frac{1}{p^s} \lra \infty \text{ as } s \lra 1^+
$$
which implies that there are infinitely many primes $ p \equiv b \mod 5 $ for $ b \in \{ 1,2,3,4 \} $.

\section*{Question 2}
Let $ \chi $ be any Dirichlet character. Then, for $ s > 1 $, prove that
$$
	\frac{1}{L(s, \chi)} = \sum_{n=1}^\infty \frac{\chi(n)\mu(n)}{n^s}
$$
\subsection*{Solution}
This is almost obvious.

Firstly we know that 
\begin{align*}
	L(s,\chi) = \prod_{p \text{ prime}} \left( 1 - \frac{\chi(p)}{p^s} \right)^{-1}
\end{align*}
so 
\begin{align*}
	\frac{1}{L(s,\chi)} &= \prod_{p \text{ prime}} \left( 1 - \frac{\chi(p)}{p^s} \right) \\
		&= \left( 1 - \frac{\chi(2)}{2^s}\right) \left( 1 - \frac{\chi(3)}{3^s}\right)\left( 1 - \frac{\chi(5)}{5^s}\right)\cdots \\
		&\text{ Peter Brown: Deep breath...} \\
		&= 1 - \frac{\chi(2)}{2^s} - \frac{\chi(3)}{3^s} - \frac{\chi(5)}{5^s} - \cdots \\
		& \ \ \ + \frac{\chi(2)\chi(3)}{2^s3^s} + \frac{\chi(2)\chi(5)}{2^s5^s} + \cdots + \frac{\chi(3)\chi(5)}{3^s5^s} + \cdots \\
		& \ \ \ - \frac{\chi(2)\chi(3)\chi(5)}{2^s3^s5^s} - \cdots \\
		& \ \ \ \ \ \vdots
\end{align*}
and because $ \chi $ is completely multiplicative
\begin{align*}
	\frac{1}{L(s,\chi)}
		&= 1 - \frac{\chi(2)}{2^s} - \frac{\chi(3)}{3^s} - \frac{\chi(5)}{5^s} - \cdots \\
		& \ \ \ + \frac{\chi(2 \times 3)}{(2\times 3)^s} + \frac{\chi(2\times 5)}{(2 \times 5)^s} + \cdots + \frac{\chi(3 \times 5)}{(3 \times 5)^s} + \cdots \\
		& \ \ \ - \frac{\chi(2 \times 3 \times 5)}{(2 \times 3 \times 5)^s} - \cdots \\
		& \ \ \ \ \ \vdots
\end{align*}
It's clear from the definition of $ \mu $ that this can be rewritten as
\begin{align*}
	\frac{1}{L(s,\chi)} = \sum_{n=1}^\infty \frac{\chi(n)\mu(n)}{n^s}.
\end{align*}

\section*{Question 3}
Suppose $ \chi_4 $ and $ \chi_6 $ are the (unique) non-principle characters modulo 4 and 6 respectively.
Show that $ L(1, \chi_4)  = \frac{\pi}{4} $ and $ L(1, \chi_6) = \frac{\pi}{2\sqrt{3}} $. 

\subsection*{Solution}
We have that for a non-principle character modulo $ k $
\begin{align*}
	L(1, \chi) &= \int_0^1 \frac{\lambda(t)}{1-t^k} dt \\
\end{align*}
where $ \lambda(t) = \sum_{n=1}^k \chi(n)t^{n-1} $.

In the case of $\chi_4 $ we have that 
$ \lambda(t) = 1 - t^2 $,
and so we have the evaluate
\begin{align*}
	\int_0^1 \frac{1-t^2}{1-t^4} dt &= \int_0^1 \frac{(1-t^2)}{(1-t^2)(1+t^2)}dt \\
		&= \left[ \tan^{-1}(t) \right]^1_1 \\
		&= \frac{\pi}{4}
\end{align*}
so $ L(1, \chi_4)  = \frac{\pi}{4} $.

In the case of $ \chi_6 $ we have that $ \lambda(t) = 1 - t^4 $, and so we evaluate
\begin{align*}
	\int_0^1 \frac{1-t^4}{1-t^6} dt &= \int_0^1 \frac{(1-t^2)(1+t^2}{(1-t^2)(1+t^2+t^4)}dt \\
		&= \int_0^1 \frac{(1+t^2)}{(1+t^2+t^4)} \\
		&= \int_0^1 \frac{(1+\frac{1}{t^2}}{(\frac{1}{t^2} + 1 + t^2)} dt.
\end{align*}
Let $ x = t - \frac{1}{t} $ and note that $ dx = (1 + \frac{1}{t^2})dt $
and note that as $ t \lra 0+ $, $ x \lra -\infty $ and when $ t = 1 $, $ x = 0 $.
So we have that 
\begin{align*}
	\int_0^1 \frac{(1+\frac{1}{t^2}}{(\frac{1}{t^2} + 1 + t^2)} dt &= \int_{-\infty}^0 \frac{1}{x^2 + 3} dx \\
		&= \left[ \frac{1}{\sqrt{3}}\tan^{-1}\left(\frac{x}{\sqrt{3}}\right)\right]_{x \lra -\infty}^0 \\
		&= \frac{\pi}{2\sqrt{3}}
\end{align*}
so $ L(1, \chi_6) = \frac{\pi}{2\sqrt{3}} $.

\section*{Acknowledgements}
Thank you to Roberto for reminding me to bound the imaginary parts of the $ L $ functions in question 1 part b.
\end{document}


