\documentclass{unswmaths}
\usepackage{color}
\usepackage{unswshortcuts}
\begin{document}
\author{Adam J. Gray}
\studentno{3329798}
\subject{Number Theory}
\title{Assignment}

\unswtitle

\section*{Question 1}
\subsection*{Part a}
Use the character table given in lectures for $ \mathbb{Z}_5 $, extended to a Dirichlet character,to evaluate
$$
	\sum_{i=1}^4 \chi_i(n) \overline{ \chi_i(b)}, \ \ \ \text{ for each } b \in \mathbb{U}_5.
$$
\subsection*{Part b}
Use the results of (a) to prove, in detail, that there are infnitely many primes congruent to  $ 1 \mod 5 $,
$ 2 \mod 5 $, and $ 3 \mod 5 $ and $ 4 \mod 5 $.

\subsection*{Solution}
\subsection*{Part a}
\begin{align*}
	\sum_{i=1}^4 \chi_i(n) \overline{ \chi_i(b)} &=
	\begin{cases}
		0 & \text{ if } n \not\equiv b \mod 5 \\
		4 & \text{ if } n \equiv b \mod 5
	\end{cases}
\end{align*}
This follows immediatly from the orthogonality relation proved in lectures.
\subsection*{Part b}
\end{document}


