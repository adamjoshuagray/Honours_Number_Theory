\documentclass{unswmaths}
\usepackage{color}
\usepackage{unswshortcuts}
\begin{document}
\author{Adam J. Gray}
\studentno{3329798}
\subject{Number Theory}
\title{Assignment}

\unswtitle

\section*{Question 1}
\subsection*{Part a}
Use the character table given in lectures for $ \mathbb{Z}_5 $, extended to a Dirichlet character,to evaluate
$$
	\sum_{i=1}^4 \chi_i(n) \overline{ \chi_i(b)}, \ \ \ \text{ for each } b \in \mathbb{U}_5.
$$
\subsection*{Part b}
Use the results of (a) to prove, in detail, that there are infnitely many primes congruent to  $ 1 \mod 5 $,
$ 2 \mod 5 $, and $ 3 \mod 5 $ and $ 4 \mod 5 $.

\subsection*{Solution}
\subsection*{Part a}
\begin{align*}
	\sum_{i=1}^4 \chi_i(n) \overline{ \chi_i(b)} &=
	\begin{cases}
		0 & \text{ if } n \not\equiv b \mod 5 \\
		4 & \text{ if } n \equiv b \mod 5
	\end{cases}
\end{align*}
This follows immediatly from the orthogonality relation proved in lectures.
\subsection*{Part b}
Firstly see that 
\begin{align*}
	L(s, \chi_i) &= \sum_{n=1}^\infty \frac{\chi_i(n)}{n^s} \\
		&= \prod_{p \text{ prime }} \left( 1 - \frac{\chi_i(p)}{p^s} \right)^{-1}
\end{align*}
and so 
\begin{align*}
	\log L(s, \chi_i) &= - \sum_{n=1}^\infty \log \left( 1 - \frac{\chi_i(p)}{p^s} \right) \\
		&= \sum_{p \text{ prime}} \frac{\chi_i(p)}{p^s} + \frac{1}{2}\left( \frac{\chi_i(p)}{p^s} \right)^2 
			+ \frac{1}{3}\left( \frac{\chi_i(p)}{p^s} \right)^3 + \cdots \\
		&= \sum_{p \text{ prime}} \frac{\chi_i(p)}{p^s} + R_i(s)
\end{align*}
where
\begin{align*}
	R_i(s) &= \sum_{p \text{ prime}} \frac{1}{2}\left( \frac{\chi_i(p)}{p^s} \right)^2 + 
		\frac{1}{3}\left( \frac{\chi_i(p)}{p^s} \right)^3 + \frac{1}{4}\left( \frac{\chi_i(p)}{p^s} \right)^4 + \cdots. \\
\end{align*}
See that
\begin{align*}
	|R_i(s)| &\leq \sum_{p \text{ prime}} \frac{1}{2}\left( \frac{|\chi_i(p)|}{p^s} \right)^2 + 
		\frac{1}{3}\left( \frac{|\chi_i(p)|}{p^s} \right)^3 + \frac{1}{4}\left( \frac{|\chi_i(p)|}{p^s} \right)^4 + \cdots \\
		&\leq \sum_{p \text{ prime}} \frac{1}{2}\left( \frac{1}{p^s} \right)^2 + 
		\frac{1}{2}\left( \frac{1}{p^s} \right)^3 + \frac{1}{2}\left( \frac{1}{p^s} \right)^4 + \cdots \\
		& \text{ by the geometric sum formula } \\
		&= \frac{1}{2} \sum_{p \text{ prime}} \frac{1}{p^s(p^s-1)} \\
		&< \frac{1}{2} \sum_{p \text{ prime}} \frac{1}{p(p-1)} \\
		&< \frac{1}{2} \sum_{n=1}^\infty \frac{1}{n(n-1)} \\
		&= \frac{1}{2}
\end{align*}
and so $ R_i(s) $ is bounded as $ s \longrightarrow 1^{+} $.

\end{document}


