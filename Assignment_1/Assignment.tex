\documentclass{unswmaths}
\usepackage[a4paper]{geometry}
\usepackage{fancyhdr}
\pagestyle{fancy}
\begin{document}

\setlength\parindent{0pt}

\unswtitle{Adam J. Gray}{3329798}{Number Theory}{Assignment}
\fancyfoot[l]{Adam J. Gray}
\fancyfoot[r]{\today}
\fancyhead[l]{The University of New South Wales}
\fancyhead[r]{Number Theory}


\section*{Question 1}
\subsection*{Part a}
Prove the following theorem.
\begin{unswthm}
Let n be an integer for which $ \mathbb{U}_n $ admits primitive roots and 
suppose $ (a,n) = 1 $. Then the congruence  $ x^k \equiv a \mod n $  has a solution 
if and only if $$ a^\frac{\phi(n)}{d} \equiv 1 \mod n $$
where $ d= (k, \phi(n)) $. Furthermore, if it has solutions, then it has exactly d
solutions in $ \mathbb{U}_n $.
\end{unswthm}
\begin{proof}
Let $ g \in \mathbb{U}_{n} $ be a primitive root.  
Now as $ (a, n) = 1 $ and $ g $ is a primitive root, there must exist an $ r $
such that $$ g^r \equiv a \mod n. $$

Now it suffices to consider $ x \in \{ g^1, g^2, \ldots, g^{\phi(n)} \} $ because if
$ (x, n) \neq 1 $ then $ (x^k, n) \neq 1 $ for all $ k $, which conflicts with
$ (a, n) = 1 $. 

So there exists an $ s $ such that $ g^s \equiv x \mod n $. We can now write
\begin{align}
	&&x^k &\equiv a \mod n \nonumber \\
	&\Leftrightarrow &g^{sk} &\equiv g^r \mod n \nonumber \\
	\label{eq:Mod_Equiv}
	&\Leftrightarrow &sk &\equiv r \mod \phi(n)
\end{align}

Now \eqref{eq:Mod_Equiv} has a solution if and only if $ (k, \phi(n)) \mid r $,
that is if and only if $ d \mid r $. Also if a solution exists there are clearly
$ d $ such solutions.

We now wish to show $ d \mid r \Leftrightarrow a^\frac{\phi(n)}{d} \equiv 1 \mod n $.
Note that 
\begin{align*}
	a^\frac{\phi(n)}{d} & \equiv g^{\phi(n)\frac{r}{d}} \mod n
\end{align*}
and because 
$ \operatorname{ord}_n(g) = \phi(n) $
we have that
$$ g^{\phi(n)\frac{r}{d}} \equiv 1 \mod n, $$
if and only if $ d \mid r $.
That is to say $ a^\frac{\phi(n)}{d} \equiv 1 \mod n $ if and only if $ d \mid r $.

So we have shown that a solution to $$ x^k \equiv a \mod n $$ exists if and only if
$$ a^\frac{\phi(n)}{d} \equiv 1 \mod n $$
and that if a solution exists, there are exactly $ d $ solutions.
\end{proof}

\subsection*{Part b}
Prove the following lemma.
\begin{unswlem}
If $ p $ is a prime of the form $ 6k - 1 $, then $ x^3 \equiv a \mod p $ 
has a unique solution for every integer $ a $.
\end{unswlem}
\begin{proof}
See that 
\begin{align}
	\label{eq:gcd}
	(\phi(p), 3) 	&= (6k - 2, 3) \\
			&= 1. \nonumber
\end{align}
From the lemma above we have that $ x^3 \equiv a \mod p $ has $ d $ solutions if
$$ a^\frac{\phi(p)}{d} \equiv 1 \mod n $$ where $ d =  (\phi(p), 3 ) $.
The uniqueness of the solution is therefore guaranteed by \eqref{eq:gcd}.

We just have to show that $ a^{\phi(p)} \equiv 1 \mod p $ for all $ a $, 
but Euler's theorem guarantees that this must be the case for $ (a, p) = 1 $. 
As $ p $ is prime this is true for all $ a \in \mathbb{Z}_p $, except $ 0 $.
So for all $ a \in \mathbb{Z}_p $, $ a \not\equiv 0 \mod p $ there is a solution.
When $ a \equiv 0 \mod p $ it is clear that $ x \equiv 0 \mod p $ is a solution.  
So we have shown that for primes of the form $ p = 6k + 1 $,
$ x^3 \equiv a \mod p $ has a \emph{unique} solution for all $ a $.
\end{proof}

\section*{Question 2}
Suppose $ q $ and $ p = 2q + 1 $ are both prime, with $ q > 2 $. Prove that $ 2q $ is the 
only element in $ \mathbb{Z}_p $ which is a quadratic non-residue, 
but not a primitive root. Hence find all the primitive roots in $ \mathbb{Z}_{23} $. 
\subsubsection*{Solution}
If $ g $ is a primitive root $ \mod p $ then it must be that 
$ g^{\frac{p-1}{2}} \equiv -1 \mod p $, but Euler's criterion tells us that $ g $ must
therefore be a a quadratic non-residue. That is to say, all primitive roots must be
quadratic non-residues. 

The number of primitive roots of $ \mathbb{Z}_p $ is 
\begin{align}
\phi(\phi(p)) 	&= \phi(\phi(2q+1)) \nonumber \\
		&= \phi(2q) \nonumber \\
		&= \phi(2)\phi(q) \nonumber \\
		&= q - 1. \label{eq:num_pr}
\end{align}
From the lectures it is know that the number of quadratic residues (and hence
non-residues) $ \mod p $ is 
\begin{equation}
\label{eq:num_qr}
\frac{p - 1}{2} = q.
\end{equation}

So from \eqref{eq:num_pr} and \eqref{eq:num_qr} we can conclude that there must 
be precisely one quadratic non-residue which is not a primitive root in $ \mathbb{Z}_p $.

Consider $ 2q $ and note that $ 2q \equiv -1 \mod p $ so, $ 2q $ cannot possibly be a 
primitive root.
We can also see that
\begin{align*}
	(-1)^{\frac{p-1}{2}} 	&\equiv (-1)^{q} \\
				&\equiv (-1)
\end{align*}
as we have $ q > 2 $, so by Euler's criterion we $ 2q $ must be a quadratic non-residue.

So we have shown that $ 2q $ is the only quadratic non-residue in $ \mathbb{Z}_{p} $.

We now consider $ \mathbb{Z}_{23} $ and note that $ 23 = 11 * 2 + 1 $ and that $ 11 $ is
a prime. This means we can apply the above result to list all the primitive roots in 
$ \mathbb{Z}_{23} $.
They are $ \{ 5,7,10,11,14,15,17,19,20,21\} $.
\section*{Question 3}
\subsection*{Part a}
Show that for any positive integer $ n $, the number $ 4n + 2 $ is the sum of three 
squares, exactly two of which are odd.


\subsubsection*{Solution}
From the lectures it is known that a number, $ m $, can be written as the sum of three
squares if and only if it cannot be written in the form $ m = 4^\alpha(8k + 7 ) $, 
where $ \alpha, k \in \mathbb{Z} $. So if we let $ m = 4n + 2 $ it suffices to show 
that $ 4 \nmid  m $ and $ m \not\equiv 7 \mod 8 $. 

Clearly $ 4n + 2 \equiv 2 \text{ or } 6 \mod 8 $. Additionally $ 4n + 2 \equiv 2 \mod 4 $.
So $ m \not\equiv 7 \mod 8 $ and $ 4 \nmid m $, therefore it must be possible to write $ m $ 
as the sum of three squares.

Now we wish to show that two of the squares must be odd. Note that for even $ x $,
we must have $ x^2 \equiv 0 \mod 4 $ and for odd $ x $ we must have 
$ x^2 \equiv 1 \mod 4 $. Then if we have $$ m \equiv 2 \equiv x^2 + y^2 + z^2 \mod 4 $$ we must
have that exactly two of $ \{ x^2, y^2, z^2 \} $ are congruent $ 1 \mod 4 $, that is, two of 
the numbers must be odd. 
\qed 

\subsection*{Part b}
Deduce that every odd positive integer can be expressed in the form $ a^2 + b^2 + 2c^2 $.
\subsubsection*{Solution}
From the above statement we know that for any positive $ n \in \mathbb{Z} $, 
$ 4n + 2 = x^2 + y^2 + z^2 $, for some positive integers $ \{ x, y, z \} $, and that exactly
one of these integers is even. Without loss of generality suppose $ z $ is even.  Then
\begin{align*}
	2(2n + 1) &= x^2 + y^2 +z^2 \\
	2n + 1 &= \frac{x^2 + y^2}{2} + 2c^2 \\
\end{align*}
where $ c = \frac{z}{2} $.

Let $$ d = \frac{x^2 + y^2}{2}, $$ where $ x \text{ and } y $ are odd, so that it suffices to show that $ d $ can be written as 
the sum of two integer squares.
Note that 
\begin{align*}
	d 	 	    &= \frac{x^2 + y^2}{2} \\
			    &= 	{\underbrace{\left( \frac{x + y}{2} \right)}_{\in \mathbb{Z}}}^2 + 
		              	{\underbrace{\left( \frac{x - y}{2} \right)}_{\in \mathbb{Z}}}^2
\end{align*}
and since $ x \text{ and } y $ are both odd, $ \frac{x+y}{2} \text{ and } \frac{x-y}{2} $ are both integers.
Letting 
$$ a = \frac{x+y}{2} $$
and
$$ b = \frac{x-y}{2} $$
we have shown that we can write $ d = a^2 + b^2 $ and hence have shown that we can write
$$ 2n + 1 = a^2 + b^2 + 2c^2 $$ for some integers, $ \{ a, b, c \} $ and $ n $ a 
positive integer. 

For completeness it remains only to tie up one edge case when we wish to write 1 as the sum of three squares, 
but this is clearly achieved with $ a = 1, b = 0, c = 0 $. 
\subsection*{Acknowledgements}
Thank you to Kent Yang for mentioning a technique which reduced a larger proof by cases 
argument in Question 3, Part a, to a single, short, modulus argument.

Thank you to Roberto Riedig for pointing out an extra case to consider in my solution for Question 1, Part b.

\end{document}

